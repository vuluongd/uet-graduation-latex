\documentclass{uetgraduation}
\begin{document}
\studentname{Vũ Đức Lương}
\title{TÍNH TOÁN TỐI ƯU QUÁ TRÌNH CHUYỂN ĐỔI ĐỘI HÌNH DRONE TRONG TRÌNH DIỄN ÁNH SÁNG}
\documenttype{Đồ án tốt nghiệp hệ đại học chính quy}
\major{Công nghệ hàng không vũ trụ}
\year{2025}
\supervisor{TS. Nguyễn Hoàng Quân}
\makecovers
\begin{preamble}{Tóm tắt}
\textbf{Tóm tắt: } Trình diễn ánh sáng bằng drone là một hình thức biểu diễn trong đó hàng trăm hàng nghìn drone bay theo lộ trình đã được lập sẵn để tạo ra những hình dáng đội hình khác nhau trên bầu trời, tạo nên những hình ảnh và hiệu ứng thị giác ấn tượng. Công nghệ này ngày càng được ứng dụng trong các sự kiện lớn hay các lễ hội như một giải pháp thay thế cho pháo hoa truyền thống. Trong đồ án này, thuật toán Hungarian được sử dụng để tối ưu hóa việc gán nhiệm vụ với chi phí được tính toán dựa vào khoảng cách của các drone đến vị trí mục tiêu tương ứng từ các đội hình được thiết kế sẵn, nhằm tìm phương pháp phân công tối ưu với tổng quãng đường nhỏ nhất. Để đảm bảo quá trình chuyển đổi đội hình được mượt mà và an toàn, một thuật toán quy hoạch đường đi được kết hợp với thuật toán trường thế năng nhân tạo (Artificial Potential Field-APF) được kết hợp với thuật toán hungarian nhằm có thể tính toán quỹ đạo di chuyển của từng drone trong quá trình chuyển đổi đội hình. Thuật toán trường thế năng nhân tạo được cải tiến với trường đẩy được phân lớp để giảm dao động cho drone khi di chuyển, đông thời các ràng buộc động học cũng được thêm vào để tăng độ ổn định và an toàn cho drone trong quá trình chuyển động. Bên cạnh đó thuật toán hoán đổi mục tiêu cũng được tích hợp nhằm giúp các drone có thể tránh được việc bị kẹt trong các điểm cục bộ trong quá trình di chuyển giúp các drone đều có thể hoàn thành được nhiệm vụ.Ngoài ra, đồ án cũng triển khai tăng tốc tính toán bằng GPU thông qua PyOpenCL nhằm giảm thời gian xử lý cho thuật toán quy hoạch đường đi 3D trong các mô phỏng chuyển đổi đội hình với số lượng drone lớn. Các kết quả mô phỏng về khoảng cách giữa các drone, vận tốc, gia tốc và thời gian bay, cho thấy phương pháp đề xuất hoạt động hiệu quả và có tính ứng dụng cao cho các đội hình drone quy mô lớn.     

\textit{\textbf{Từ khóa:} Drone trình diễn ánh sáng, thuật toán Hungarian, quy hoạch đường đi, thuật toán trường thế năng nhân tạo, GPU, PyOpenCL}
\end{preamble}
\begin{preamble}{Lời cam đoan} 
    \indent Tôi tên là Vũ Đức Lương, sinh viên lớp K66AE, Viện Công nghệ Hàng không Vũ trụ, Trường Đại học Công nghệ - Đại học Quốc gia Hà Nội.

    \indent Tôi xin cam đoan rằng đồ án tốt nghiệp với đề tài "Tính toán tối ưu quá trình chuyển đổi đội hình drone trong trình diễn ánh sáng" là công trình nghiên cứu của tôi. Mọi sự giúp đỡ trong quá trình thực hiện đồ án đã được cảm ơn, các thông tin trích dẫn trong đồ án này đều được ghi rõ nguồn tham khảo theo đúng quy định. Các số liệu nghiên cứu và kết quả trình bày trong đồ án là trung thực.

    \indent Nếu có bất kỳ sai phạm nào, tôi xin hoàn toàn chịu trách nhiệm.

    % Dòng ngày tháng căn phải hoàn toàn
    \begin{flushright}
        Hà Nội, ngày .... tháng .... năm 2025
    \end{flushright}

    \vspace{0.2cm}

% Khối chữ ký rộng 6cm, căn giữa dòng "Sinh viên" nhưng đặt về bên phải
    \noindent\hfill
    \begin{minipage}{6cm}
        \centering
        Sinh viên \\[1.5cm] % khoảng trống để ký
        \textbf{Vũ Đức Lương}
    \end{minipage}
\end{preamble}   
\begin{preamble}{Lời cảm ơn}
    \indent Trước hết, em xin gửi lời cảm ơn chân thành và sâu sắc nhất đến thầy giáo hướng dẫn là TS. Nguyễn Hoàng Quân và KS. Trần Đăng Huy đã tận tình hướng dẫn, giúp đỡ em trong quá trình thực hiện đồ án tốt nghiệp này. Bên cạnh đó em cũng gửi lời cảm ơn đến Viện Công nghệ Hàng không Vũ trụ và Trường Đại học Công nghệ đã cho em một môi trường học tập tốt với các cán bộ giảng viên có chuyên môn cao và tận tình trong công tác giảng dạy.

    \indent Em cũng xin lời cảm ơn đến gia đình và bạn bè đã luôn động viên, giúp đỡ em trong suốt quá trình học tập và thực hiện đồ án tốt nghiệp.

    \indent Mặc dù đã cố gắng hết sức, nhưng do hạn chế về kiến thức và thời gian, đồ án không tránh khỏi những thiếu sót. Em rất mong nhận được sự góp ý từ thầy cô và các bạn trong hội đồng bảo vệ để hoàn thiện hơn trong những nghiên cứu sau này.
\end{preamble}
\begin{contentlisting}
\tableofcontents
\listoffigures
\listoftables
    \begin{contentlistingsection}{Các từ viết tắt}
        APF: Artificial Potential Field -- Thuật toán trường thế năng nhân tạo.
        
        GPU: Graphics Processing Unit -- Bộ xử lý đồ họa.
        
        CPU: Central Processing Unit -- Bộ xử lý trung tâm.
        
        UAV: Unmanned Aerial Vehicle -- Máy bay không người lái.
        
        PyOpenCL : Python OpenCL -- Thư viện Python cho OpenCL.
    \end{contentlistingsection}
\end{contentlisting}
\chapter{Mở đầu}
\section{Giới thiệu về drone trình diễn ánh sáng}
Trong những năm gần đây, trình diễn ánh sáng bằng drone (drone light show) đã trở thành một xu hướng mới trong lĩnh vực giải trí và sự kiện. Công nghệ này sử dụng hàng trăm hàng nghìn UAV được lập trình để bay theo các đội hình và tạo ra những hình ảnh, hiệu ứng ánh sáng ấn tượng trên bầu trời. Nhờ vào khả năng tái sử dụng, không gây khó bụi và có tính linh hoạt cao cũng như mức độ an toàn tốt hơn mà việc sử dụng drone trong các màn trình diễn ánh sáng ngày càng được sử dụng rộng rãi đặc biệt trong những sự kiến lớn thay thế dần cho pháo hoa truyền thống.


\end{document}
