\documentclass{uetgraduation}
\begin{document}
\studentname{Vũ Đức Lương}
\title{TÍNH TOÁN TỐI ƯU QUÁ TRÌNH CHUYỂN ĐỔI ĐỘI HÌNH DRONE TRONG TRÌNH DIỄN ÁNH SÁNG}
\documenttype{Đồ án tốt nghiệp hệ đại học chính quy}
\major{Công nghệ hàng không vũ trụ}
\year{2025}
\supervisor{TS. Nguyễn Hoàng Quân}
\makecovers
\begin{preamble}{Tóm tắt}
\textbf{Tóm tắt: } Trình diễn ánh sáng bằng drone là một hình thức biểu diễn trong đó hàng trăm hàng nghìn drone bay theo lộ trình đã được lập sẵn để tạo ra những hình dáng đội hình khác nhau trên bầu trời, tạo nên những hình ảnh và hiệu ứng thị giác ấn tượng. Công nghệ này ngày càng được ứng dụng trong các sự kiện lớn hay các lễ hội như một giải pháp thay thế cho pháo hoa truyền thống. Trong đồ án này, thuật toán Hungarian được sử dụng để tối ưu hóa việc gán nhiệm vụ với chi phí được tính toán dựa vào khoảng cách của các drone đến vị trí mục tiêu tương ứng từ các đội hình được thiết kế sẵn, nhằm tìm phương pháp phân công tối ưu với tổng quãng đường nhỏ nhất. Để đảm bảo quá trình chuyển đổi đội hình được mượt mà và an toàn, một thuật toán quy hoạch đường đi được kết hợp với thuật toán trường thế năng nhân tạo (Artificial Potential Field-APF) được kết hợp với thuật toán hungarian nhằm có thể tính toán quỹ đạo di chuyển của từng drone trong quá trình chuyển đổi đội hình. Thuật toán trường thế năng nhân tạo được cải tiến với trường đẩy được phân lớp để giảm dao động cho drone khi di chuyển, đông thời các ràng buộc động học cũng được thêm vào để tăng độ ổn định và an toàn cho drone trong quá trình chuyển động. Bên cạnh đó thuật toán hoán đổi mục tiêu cũng được tích hợp nhằm giúp các drone có thể tránh được việc bị kẹt trong các điểm cục bộ trong quá trình di chuyển giúp các drone đều có thể hoàn thành được nhiệm vụ.Ngoài ra, đồ án cũng triển khai tăng tốc tính toán bằng GPU thông qua PyOpenCL nhằm giảm thời gian xử lý cho thuật toán quy hoạch đường đi 3D trong các mô phỏng chuyển đổi đội hình với số lượng drone lớn. Các kết quả mô phỏng về khoảng cách giữa các drone, vận tốc, gia tốc và thời gian bay, cho thấy phương pháp đề xuất hoạt động hiệu quả và có tính ứng dụng cao cho các đội hình drone quy mô lớn.     
\end{preamble}
\end{document}
